\documentclass{article}

\usepackage{hyperref}

\title{Quality Threshold Clustering}
\author{Loconte Lorenzo, M. 683124}

\begin{document}
	\maketitle
	\section{Introduzione}
	In statistica, il \textbf{clustering} è un insieme di tecniche volte
	alla selezione e raggruppamento di elementi omogenei in un insieme di
	dati. Le tecniche di clustering si basano su misure relative alla
	somiglianza degli elementi. Gli algoritmi di clustering raggruppano gli
	elementi sulla base della loro distanza reciproca.

	\section{QT-Clustering}
	\textbf{QT-Clustering} è composto da due applicazioni, un client ed un
	server. Lo scopo del progetto è permettere a molteplici client di
	eseguire un algoritmo di clustering su un server utlizzando come
	sorgente dei dati un database MySQL.

	\subsection{Server}
	Il server si occupa principalmente di accettare le richieste dei client.
	Esso esegue l'algoritmo di clustering vero e proprio utilizzando come
	sorgente dei dati una tabella (specificata dal client) di un database
	MySQL. Le funzionalità del server sono le seguenti:
	\begin{itemize}
		\item Capacità di gestire le richieste di molteplici client in
		contemporanea.
		\item Leggere il contenuto di una tabella di una base di dati
		indipendentemente dal numero di attributi e dal numero di tuple.
		\item Eseguire l'algoritmo di \textbf{Quality Threshold
		clustering} usando un raggio arbitrario su un numero indefinito
		di tuple.
		\item Fare operazioni di \textbf{Principal Component Analysis}
		per estrarre dalle tuple le componenti principali.
		\item Salvare su file l'esito della computazione come cache.
		\item Restituire ai client l'esito della computazione.
	\end{itemize}

	\subsection{Client}
	Il client è fornito di una interfaccia grafica che permette ad un'utente
	di connettersi al server e selezionare la fonte dei dati su cui eseguire
	l'algoritmo di clustering. Inoltre il client si occupa di rappresentare
	in forma grafica i cluster o i centroidi all'interno di un grafico
	bidimensionale.

	\subsection{Estensione}
	L'estensione principale rispetto al progetto originale è la grafica per
	l' applicazione client. Un'altra estensione, sempre a supporto della
	grafica, è costituita da un'algoritmo di \textbf{Principal Component
	Analysis}.

	\subsubsection{Grafica}
	La grafica per l'applicazione client è stata scritta utilizzando le
	\textbf{JavaFX}. L'interfaccia è composta da quattro sezioni:
	\begin{enumerate}
		\item La sezione per la connessione al server. L'utente
		inserisce ip e porta del server in ascolto.
		\item La sezione che contiene i parametri per il clustering,
		ovvero la tabella del database da cui ottenere i dati e il
		raggio utilizzato.
		\item La sezione che mostra il risultato della computazione,
		costituita da una \textit{TextArea}.
		\item La sezione che mostra uno scatter plot del risultato della
		computazione, costituito da uno \textit{ScatterChart} compreso
		di etichette e simboli diversi per ogni cluster. L'utente può
		scegliere, tramite pulsanti, su quale piano (XY, YZ, o XZ)
		proiettare i dati per rappresentarli sullo scatter plot.
	\end{enumerate}

	\subsubsection{Principal Component Analysis}
	La \textbf{Principal Component Analysis} (da adesso abbreviato con
	\textbf{PCA}) è una tecnica appartenente alla statistica multivariata
	usata per estrarre da un insieme di dati N dimensionali M componenti
	principali in modo tale da ridurre il numero di dimensioni dello spazio
	in cui si trovano i dati. L'esigenza di fare ciò nasce dal fatto che è
	molto difficile rappresentare tuple in uno spazio a più di tre
	dimensioni. Pertanto, tramite algoritmi di \textbf{PCA}, è possibile
	proiettare i dati in uno spazio tridimensionale (o anche bidimensionale)
	che sia trattabile e soprattuto rappresentabile. Ovviamente la scelta
	degli assi di proiezione non è arbitraria, infatti punti molto distanti
	tra loro in uno spazio N dimensionale possono ritrovarsi ad essere molto
	vicini in uno spazio M dimensionale (con M \textless N). Generalmente
	si scelgono gli assi di proiezione in modo tale che il campione
	proiettato abbia varianza massima. L'algoritmo di \textbf{PCA}
	implementato è il seguente:
	\begin{enumerate}
		\item Si trasforma il campione misto (numerico e discreto) in
		un campione numerico.
		\item Si standardizza il campione.
		\item Si calcola la matrice di covarianza del campione.
		\item Si calcolano autovalori ed autovettori della matrice di
		covarianza.
		\item Si scelgono come assi di proiezione gli autovettori
		associati agli autovalori più grandi.
		\item Si proietta il campione standardizzato sugli assi di
		proiezione.
	\end{enumerate}
	Si tenga presente che questa procedura sacrifica parte dell'informazione
	del campione originario. Il calcolo delle componenti principali è
	effettuato dal server al termine dell'esecuzione dell'algoritmo di
	clustering. E' compito del client poi disegnare sullo scatter-plot
	i punti. Come libreria per l'algebra lineare è stata usata la libreria
	\href{http://la4j.org}{\textbf{la4j}}.

	\section{Note tecniche}
	Per il progetto è stato utilizzato lo strumento \textbf{git}. Inoltre
	il progetto è stato hostato su un repository remoto privato su
	\href{https://github.com}{\textbf{GitHub}}. Come strumento di build
	system e gestione delle dipendenze è stato utilizzato \textbf{gradle}.
	In questo modo si ha la completa indipendenza dall'IDE che si intende
	utilizzare. 
	Durante lo sviluppo sono stati utilizzati i seguenti strumenti:
	\begin{itemize}
		\item \textit{JUnit 4.12}, per i casi di test.
		\item \textit{JaCoCo}, per la copertura dei casi di test.
		\item \textit{Checkstyle}, per verificare lo stile.
	\end{itemize}
	Sono stati scritti casi di test esclusivamente per il \textit{server}
	dato che il client è costituito principalmente da codice per la UI.
	I reports dei casi di test con \textit{JUnit} e di \textit{JaCoCo} sono
	visualizzabili in \textit{.../qt-clustering/server/build/reports}.
	Attualmente la copertura dei casi di test è pari al \textbf{74\%}.

	\subsection {Importare il progetto con Eclipse}
	Entrambi i progetti per il client e per il server contengono un file
	chiamato \textit{build.gradle}. E' possibile importare il progetto con
	Eclipse seguendo i seguenti passi:
	\begin{enumerate}
		\item Importare il progetto come progetto \textit{gradle}:
			\textit{File \textgreater Import ... \textgreater Gradle
			\textgreater Existing Gradle Project}
		\item Selezionare la root directory del progetto
		\item Cliccare su \textit{Finish}
	\end{enumerate}
	Verranno così importati due progetti, uno per il \textit{server} e uno
	per il \textit{client}. Gradle scaricherà automaticamente tutte le
	dipendenze necessarie. Per eseguire il \textit{client/server} basterà
	aprire la finestra dei \textit{Gradle Tasks}, selezionare il progetto,
	e cliccare su \textit{application \textgreater run}. Se la finestra
	\textit{Gradle Tasks} non è presente allora è necessario aprirla
	selezionando \textit{Window \textgreater Show View \textgreater Other...
	\textgreater Gradle \textgreater Gradle Tasks}.
\end{document}

